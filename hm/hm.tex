\documentclass[a4paper, 16pt]{article}

\usepackage[margin=3cm]{geometry}
\usepackage[utf8]{inputenc}
\usepackage[ngerman]{babel}
\usepackage[autostyle=true,german=quotes]{csquotes}
\usepackage{amsmath}
\usepackage{amssymb}
\usepackage{graphicx}
\usepackage{hyperref}
\usepackage{tikz}
\usepackage{multirow}
\usepackage{array}
% TODO: use \SI everywhere
\usepackage[binary-units=true, per-mode=symbol]{siunitx}
\usetikzlibrary{
	calc
}

\newcommand{\cmi}{\mathrm{i}}
\DeclareMathOperator{\sgn}{sgn}

\author{Paul Brinkmeier}
\title{Cheatsheet für HM I und HM II}

\begin{document}
	\maketitle
	\newpage
	\tableofcontents
	\newpage

	\section{Ableitung}

	\begin{equation*}
		(\log{x})' = \frac{1}{x}
	\end{equation*}

	\subsection{Trigonometrie}

	\begin{equation*}
		(\tan{x})' = \frac{1}{\cos^2{x}}
	\end{equation*}

	\section{Integration}

	Sei $b > a$. Dann gilt:

	\begin{equation*}
		\int\limits_{-b}^{-a}{f(x)dx} = -\int\limits_{a}^{b}{f(x)dx}
	\end{equation*}

	\subsection{Partielle Integration}

	\begin{equation*}
		\int\limits_{a}^{b}{f' \cdot g} = \left[f \cdot g\right]_{a}^{b} - \int\limits_{a}^{b}{f \cdot g'}
	\end{equation*}

	\section{Identitäten}

	\subsection{Trigonometrie}

	\begin{eqnarray*}
		&\cos^2 x + \sin^2 x = 1 \\
		\Leftrightarrow &\cos^2 x = 1 - \sin^2 x \\
		\Leftrightarrow &\sin^2 x = 1 - \cos^2 x
	\end{eqnarray*}

	\section{Grenzwerte}

	\subsection{Gegen 0}

	\begin{equation*}
		\lim\limits_{n \to 0}{\frac{\sin x}{x}} = 1
	\end{equation*}

	\subsection{Gegen $\infty$}

	\begin{equation*}
		\lim\limits_{n \to \infty}{\left(1 + \frac{k}{n}\right)^n} = e^k
	\end{equation*}

	\section{Folgen}

	\subsection{Folgen, die man kennen sollte}

	\section{Reihen}

	\subsection{Reihen, die man kennen sollte}

	\subsubsection{Geometrische Reihe}

	Sei $|k| < 1$. Dann gilt:
	\begin{equation*}
		\sum\limits_{n = 0}^{\infty}{k^n} = \frac{1}{1 - k}
	\end{equation*}

	Hierbei muss man immer beachten, dass die geometrische Reihe bei $n = 0$ anfängt.
	In vielen Aufgaben ist nach $\sum_{n = 1}^{\infty}{k^n}$ gefragt:
	
	\begin{equation*}
		\sum\limits_{n = 1}^{\infty}{k^n} = -1 + \sum\limits_{n = 0}^{\infty}{k^n},
	\end{equation*}

	da $k^0 = 1$.

	\section{Komplexe Zahlen}

	\subsection{Wurzel einer komplexen Zahl}

	\begin{equation*}
		\sqrt{z} = \sqrt{a + b\cmi} = \pm \left(\sqrt{\frac{|z| + a}{2}} + \cmi \cdot \sgn{b} \cdot \sqrt{\frac{|z| - a}{2}} \right)
	\end{equation*}
\end{document}
